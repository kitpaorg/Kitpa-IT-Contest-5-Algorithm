진흥이가 있는 바다 위에 $R \times C$ 크기의 격자가 있습니다. 격자의 위에서부터 $i$번째 행, 왼쪽에서부터 $j$번째 열에 위치한 칸을 $(i, j)$라고 정의합시다. 진흥이는 $(1, 1)$에서 출발해 $(R, C)$에 도착하려고 합니다. 진흥이는 \textbf{오른쪽이나 아래쪽으로 한 번에 한 칸씩만} 이동할 수 있습니다.

격자의 일부 칸에는 보물이 묻혀 있습니다. 보물이 있는 칸은 총 $N$개이며, 각 $i=1,2,\ldots,N$에 대하여 칸 $(x_i, y_i)$에 $v_i$개의 보물이 묻혀 있습니다. 진흥이가 보물이 묻혀 있는 칸에 도착하면 그 칸의 보물을 모두 얻을 수 있습니다. 

또한 격자에는 $M$개의 위험 구역이 있습니다. 각 위험 구역은 네 개의 정수 $(u_i, l_i, d_i, r_i)$로 표현되며, 이는 진흥이가 직사각형 범위 $u_i \le x \le d_i$, $l_i \le y \le r_i$에 속하는 칸 $(x, y)$를 지나갈 수 없음을 의미합니다. 이때 위험 구역이 진흥이의 시작점 $(1, 1)$과 도착점 $(R, C)$를 포함하지 않음이 보장됩니다. 

진흥이는 이 격자에서 최대한 많은 보물을 얻고 부자가 된 다음, 하루 종일 청소년 IT경시대회 문제를 풀며 노는 인생을 살고 싶어합니다. 진흥이가 $(R,C)$에 도착할 수 있는지 판단하고, 도착할 수 있다면 진흥이가 가져갈 수 있는 \textbf{보물의 최대 개수}를 구해 주세요.