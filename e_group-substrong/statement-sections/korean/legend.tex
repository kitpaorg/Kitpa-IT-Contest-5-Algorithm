`\texttt{0}'과 `\texttt{1}'만으로 이루어진 \textbf{길이 $1$ 이상의} 문자열 $A$가 다음 조건 중 하나를 만족하는 경우, 이러한 $A$를 \textbf{그룹 문자열}이라고 부릅니다.

\begin{itemize}
\item $A$에 `\texttt{0}'이 등장하지 않거나 `\texttt{1}'이 등장하지 않습니다.
\item $A$에서 모든 `\texttt{0}'은 모든 `\texttt{1}'보다 먼저 등장합니다.
\item $A$에서 모든 `\texttt{1}'은 모든 `\texttt{0}'보다 먼저 등장합니다.
\end{itemize}

예를 들어, ``\texttt{000}'', ``\texttt{11}'', ``\texttt{00111}'', ``\texttt{110}''은 그룹 문자열이지만, ``\texttt{1011}''이나 ``\texttt{00100}''은 그룹 문자열이 아닙니다.

`\texttt{0}'과 `\texttt{1}'만으로 이루어진 문자열 $S$에 대해, $S$의 처음과 끝에서 문자를 원하는 만큼 지워 만들 수 있는 \textbf{서로 다른 그룹 문자열의 개수}를 $f(S)$라고 정의합니다. 예를 들어, $S$가 ``\texttt{10110}``일 때 만들 수 있는 그룹 문자열은 ``\texttt{0}``, ``\texttt{01}``, ``\texttt{011}'', ``\texttt{1}``, ``\texttt{10}``, ``\texttt{11}``, ``\texttt{110}``이 있습니다. 그러므로 $S$가 ``\texttt{10110}``일 때 $f(S)$의 값은 $7$입니다. 이때 ``\texttt{10}``이 $S$에 \textbf{여러 번 등장하지만} $f(S)$를 구하는 데는 \textbf{한 번만 세는 것}에 유의하세요.

여러분에게 문자열 $X$에 대한 $Q$번의 질문이 주어집니다. 초기에 문자열 $X$는 빈 문자열입니다. 각 질문은 다음과 같은 형태입니다.

\begin{itemize}
\item $c$ $k$: 문자열 $X$의 끝에 문자 $c$를 $k$개 붙입니다. 그후 $f(X)$의 값을 구합니다.
\end{itemize}

이때 질문으로 문자열 $X$에 추가된 문자는 그다음 질문이 주어질 때에도 문자열 $X$에서 지워지지 않고 남아있습니다.

여러분은 각 질문에 대해 충분히 빨리 대답할 수 있을까요?