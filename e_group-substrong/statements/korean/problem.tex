\begin{problem}{그룹 부분 문자열과 쿼리}{standard input}{standard output}{2 seconds}{1024 megabytes}

`\texttt{0}'과 `\texttt{1}'만으로 이루어진 \textbf{길이 $1$ 이상의} 문자열 $A$가 다음 조건 중 하나를 만족하는 경우, 이러한 $A$를 \textbf{그룹 문자열}이라고 부릅니다.

\begin{itemize}
\item $A$에 `\texttt{0}'이 등장하지 않거나 `\texttt{1}'이 등장하지 않습니다.
\item $A$에서 모든 `\texttt{0}'은 모든 `\texttt{1}'보다 먼저 등장합니다.
\item $A$에서 모든 `\texttt{1}'은 모든 `\texttt{0}'보다 먼저 등장합니다.
\end{itemize}

예를 들어, ``\texttt{000}'', ``\texttt{11}'', ``\texttt{00111}'', ``\texttt{110}''은 그룹 문자열이지만, ``\texttt{1011}''이나 ``\texttt{00100}''은 그룹 문자열이 아닙니다.

`\texttt{0}'과 `\texttt{1}'만으로 이루어진 문자열 $S$에 대해, $S$의 처음과 끝에서 문자를 원하는 만큼 지워 만들 수 있는 \textbf{서로 다른 그룹 문자열의 개수}를 $f(S)$라고 정의합니다. 예를 들어, $S$가 ``\texttt{10110}``일 때 만들 수 있는 그룹 문자열은 ``\texttt{0}``, ``\texttt{01}``, ``\texttt{011}'', ``\texttt{1}``, ``\texttt{10}``, ``\texttt{11}``, ``\texttt{110}``이 있습니다. 그러므로 $S$가 ``\texttt{10110}``일 때 $f(S)$의 값은 $7$입니다. 이때 ``\texttt{10}``이 $S$에 \textbf{여러 번 등장하지만} $f(S)$를 구하는 데는 \textbf{한 번만 세는 것}에 유의하세요.

여러분에게 문자열 $X$에 대한 $Q$번의 질문이 주어집니다. 초기에 문자열 $X$는 빈 문자열입니다. 각 질문은 다음과 같은 형태입니다.

\begin{itemize}
\item $c$ $k$: 문자열 $X$의 끝에 문자 $c$를 $k$개 붙입니다. 그후 $f(X)$의 값을 구합니다.
\end{itemize}

이때 질문으로 문자열 $X$에 추가된 문자는 그다음 질문이 주어질 때에도 문자열 $X$에서 지워지지 않고 남아있습니다.

여러분은 각 질문에 대해 충분히 빨리 대답할 수 있을까요?

\InputFile
첫 번째 줄에 정수 $Q$가 주어집니다.

두 번째 줄부터 $Q$개의 줄에 걸쳐 각 줄에 질문에 대응되는 문자 $c_i$와 양의 정수 $k_i$가 주어집니다.

\OutputFile
$Q$개의 줄에 걸쳐 각 줄에 질문에 대한 정답을 출력합니다.

\Scoring
\begin{itemize}
\item $1 \le Q \le 200\,000$
\item 각 $1 \le i \le Q$에 대해 $c_i$는 `\texttt{0}' 또는 `\texttt{1}' 
\item 각 $1 \le i \le Q$에 대한 $k_i$의 합은 $1\,000\,000\,000$ 이하
\end{itemize}

\textbf{서브태스크}

\begin{tabular}{|l|l|l|} \hline
  \textbf{번호} & \textbf{배점} & \textbf{제한} \\ \hline
  1 & 11 & 각 $1 \le i \le Q$에 대한 $k_i$의 합은 $100$ 이하 \\ \hline
  2 & 23 & 각 $1 \le i \le Q$에 대한 $k_i$의 합은 $3000$ 이하 \\ \hline
  3 & 37 & 각 $1 \le i \le Q$에 대한 $k_i$의 합은 $1\,000\,000$ 이하 \\ \hline
  4 & 29 & 추가 제한 없음 \\ \hline
\end{tabular}

\Example

\begin{example}
\exmpfile{example.01}{example.01.a}%
\end{example}

\Note
예제의 각 질문에 대해 새로운 문자열 $X$의 내용, 새롭게 만들 수 있는 그룹 문자열, 그리고 $f(X)$의 값은 다음과 같습니다.

\begin{tabular}{|l|l|l|l|} \hline
  \textbf{질문} & \textbf{문자열 $X$} & \textbf{새롭게 만들 수 있는 그룹 문자열} & $f(X)$ \\ \hline
  \texttt{1} $1$ & "\texttt{1}" & "\texttt{1}" & $1$ \\ \hline
  \texttt{0} $1$ & "\texttt{10}" & "\texttt{0}", "\texttt{10}" & $3$ \\ \hline
  \texttt{1} $2$ & "\texttt{1011}" & "\texttt{01}", "\texttt{011}", "\texttt{11}" & $6$ \\ \hline
  \texttt{0} $1$ & "\texttt{10110}" & "\texttt{110}" & $7$ \\ \hline
  \texttt{1} $2$ & "\texttt{1011011}" & 없음 & $7$ \\ \hline
  \texttt{1} $2$ & "\texttt{101101111}" & "\texttt{0111}", "\texttt{01111}", "\texttt{111}", "\texttt{1111}" & $11$ \\ \hline
\end{tabular}

\end{problem}

