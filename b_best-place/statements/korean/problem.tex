\begin{problem}{최고의 맛집을 찾아서}{standard input}{standard output}{2 seconds}{1024 megabytes}

진흥이가 사는 동네에는 $1$번부터 $M$번까지 총 $M$개의 식당이 있습니다.
진흥이를 포함한 $N$명의 친구들은 각 식당의 음식 맛을 $1$점부터 $5$점까지의 별점으로 평가했습니다.

이제 이 평가를 바탕으로 최고의 맛집을 정하려고 합니다.
어떤 식당이 최고의 맛집이 되려면 모든 친구에 대해, 그 친구가 해당 식당보다 더 높은 점수를 준 다른 식당이 없어야 합니다.

예를 들어, 어떤 친구가 $1$번 식당에는 $4$점을, $2$번 식당에는 $5$점을 줬다면
그 친구의 기준에서 $1$번 식당은 최고의 맛집이 될 수 없습니다. 왜냐하면 더 높은 점수를 받은 $2$번 식당이 존재하기 때문입니다.

하지만 이러한 조건을 만족하는 식당이 하나도 없을 수도 있습니다.
따라서 진흥이는 친구들이 준 별점을 일부 조작해서라도,
모든 식당이 최고의 맛집이 될 수 있도록 만들고자 합니다.

별점 조작은 한 친구가 어떤 식당에 준 점수를 다른 점수로 바꾸는 것을 의미합니다.

각 식당이 최고의 맛집이 되기 위해 필요한 최소 별점 조작 횟수를 구하세요.

\InputFile
첫 번째 줄에 양의 정수 $N$과 $M$이 공백으로 구분되어 주어집니다.

다음 $N$개의 줄에는 각 사람이 $1$번 식당부터 $N$번 식당까지 매긴 별점이 공백으로 구분되어 주어집니다. 각 별점은 $1$ 이상 $5$ 이하의 정수입니다.

\OutputFile
$M$개의 정수를 공백으로 구분하여 한 줄에 출력합니다.
이때 $i$번째 정수는 $i$번 식당이 최고의 맛집이 되기 위해 필요한 최소 별점 조작 횟수를 의미합니다.

\Scoring
\begin{itemize}
\item $1 \le N, M \le 1\,000$
\end{itemize}

\textbf{서브태스크}
\begin{tabular}{|l|l|l|} \hline
  \textbf{번호} & \textbf{배점} & \textbf{제한} \\ \hline
  1 & 11 & $N = 1$ \\ \hline
  2 & 19 & $N, M \le 200$ \\ \hline
  3 & 31 & 모든 사람은 별점을 $1$점 또는 $5$점만 줌 \\ \hline
  4 & 39 & 추가 제한 없음 \\ \hline
\end{tabular}

\Example

\begin{example}
\exmpfile{example.01}{example.01.a}%
\end{example}

\end{problem}

