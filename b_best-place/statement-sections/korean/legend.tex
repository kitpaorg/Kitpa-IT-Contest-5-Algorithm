진흥이가 사는 동네에는 $1$번부터 $M$번까지 총 $M$개의 식당이 있습니다.
진흥이를 포함한 $N$명의 친구들은 각 식당의 음식 맛을 $1$점부터 $5$점까지의 별점으로 평가했습니다.

이제 이 평가를 바탕으로 최고의 맛집을 정하려고 합니다.
어떤 식당이 최고의 맛집이 되려면 모든 친구에 대해, 그 친구가 해당 식당보다 더 높은 점수를 준 다른 식당이 없어야 합니다.

예를 들어, 어떤 친구가 $1$번 식당에는 $4$점을, $2$번 식당에는 $5$점을 줬다면
그 친구의 기준에서 $1$번 식당은 최고의 맛집이 될 수 없습니다. 왜냐하면 더 높은 점수를 받은 $2$번 식당이 존재하기 때문입니다.

하지만 이러한 조건을 만족하는 식당이 하나도 없을 수도 있습니다.
따라서 진흥이는 친구들이 준 별점을 일부 조작해서라도,
모든 식당이 최고의 맛집이 될 수 있도록 만들고자 합니다.

별점 조작은 한 친구가 어떤 식당에 준 점수를 다른 점수로 바꾸는 것을 의미합니다.

각 식당이 최고의 맛집이 되기 위해 필요한 최소 별점 조작 횟수를 구하세요.